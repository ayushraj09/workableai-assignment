\documentclass{article}
\usepackage{amsmath}
\usepackage{amssymb}
\begin{document}

\begin{enumerate}
    \item State the range of the random variable \(Y\) which takes values \(\{-1, 1\}\).
    
    \item Let \(X(w)\) be the number of tails in the sample space \(S\). Given the outcomes \(HTT\), \(THT\), \(TTH\), \(HHH\), \(HHT\), \(HTH\), \(THH\), find the values of \(X\) for each outcome and determine the range of \(X\).
    
    \item Given the sample space for coin tosses where \(X(w)\) = number of heads in \(w \in S\), and the outcomes \(\{TTT, TTH, THT, HTT, HHT, HTH, THH, HHH\}\), find the probabilities:
    \begin{itemize}
        \item \(P(X=0)\)
        \item \(P(X=1)\)
        \item \(P(X=2)\)
        \item \(P(X=3)\)
    \end{itemize}
    Verify that these probabilities sum to 1.
    
    \item State the properties of the probability distribution of a discrete random variable \(X\) with possible values \(x_1, x_2, \dots, x_n\) and corresponding probabilities \(p_1, p_2, \dots, p_n\).
    
    \item Given the probability distribution of a discrete random variable \(X\), express the cumulative probability \(P(X \leq x_i)\) in terms of \(P(X = x_j)\) for \(j=1,2,\dots,i\).
    
    \item Show that for a discrete random variable \(X\), the probability \(P(X \geq x_i)\) can be written as the sum of probabilities \(p_j\) for \(j = i, i+1, \dots, n\).
    
    \item Express the probability \(P(X > x_i)\) in terms of the probabilities \(p_j\) for \(j = i+1, i+2, \dots, n\).
    
    \item Write the relationship between the probabilities \(P(X \geq x_i)\), \(P(X > x_i)\), and the complement of the cumulative distribution functions.
    
    \item What is the probability that \(X\) lies between \(x_i\) and \(x_j\)? Express your answer in terms of the probabilities at individual points.
    
    \item (i) Derive the expression for the probability \(P(X = x_j)\) in terms of the given parameters.
    
    \item (i) Find the equation of the line passing through the point \((2,3)\) and parallel to the line \(4x - 5y + 7 = 0\).
    
    \item (ii) Find the equation of the line passing through the point \((1, -2)\) and perpendicular to the line \(3x + 4y = 0\).
    
    \item Examine the probability distribution of the random variable \(X\) with the probabilities:
    \[
    P(X=0) = 0.4, \quad P(X=1) = 0.4, \quad P(X=2) = 0.2.
    \]
    Verify whether these probabilities form a valid distribution.
    
    \item Given the probabilities:
    \[
    P(X=0) = 0.1, \quad P(X=1) = 0.5, \quad P(X=2) = 0.2, \quad P(X=3) = -0.1, \quad P(X=4) = 0.3,
    \]
    check whether this is a valid probability distribution.
    
    \item An experiment involves a random variable \(X\) defined as:
    \[
    X(w) = \begin{cases}
    1, & \text{if the outcome } w \text{ is an even number} \\
    0, & \text{if the outcome } w \text{ is an odd number}
    \end{cases}
    \]
    with equal probability for even and odd outcomes.
    
    \begin{itemize}
        \item[(i)] Determine the value of \(k\). Find \(P(X < 2)\), \(P(X \leq 2)\), and \(P(X \geq 2)\).
        \item[(ii)] Calculate \(P(X < 2) = P(X=0) + P(X=1) = k + 2k = 3k = \frac{3}{6} = \frac{1}{2}\).
        \item[(iii)] Calculate \(P(X \leq 2) = P(X=0) + P(X=1) + P(X=2) = k + 2k + 3k = 6k = 1\).
        \item[(iv)] Calculate \(P(X \geq 2) = 1 - P(X < 2) = 1 - \frac{1}{2} = \frac{1}{2}\).
    \end{itemize}
    
    \item Find the probability \(P(X \geq 2)\).
    
    \item Find the probability \(P(X \leq 2)\).
    
    \item Given the probability distribution:
    \[
    \begin{array}{|c|c|c|c|c|c|c|c|c|c|c|c|}
    \hline
    X & 0 & 1 & 2 & 3 & 4 & 5 & 6 & 7 \\
    \hline
    P(X) & 0 & k & 2k & 2k & 3k & k^2 & 2k^2 & 7k^2 + k \\
    \hline
    \end{array}
    \]
    Find:
    \begin{enumerate}
        \item \( P(X \geq 6) \)
        \item \( P(0 < X < 5) \)
        \item \( P(X < 6) \)
    \end{enumerate}
    
    \item Calculate \(P(X<6)\) given the probabilities \(P(X=0), P(X=1), P(X=2), P(X=3), P(X=4), P(X=5)\). Express \(P(X<6) = P(X=0) + P(X=1) + P(X=2) + P(X=3) + P(X=4) + P(X=5)\). Given:
    \[
    P(X=0) = k, \quad P(X=1) = 2k, \quad P(X=2) = 2k, \quad P(X=3) = 3k, \quad P(X=4) = k, \quad P(X=5) = k^2.
    \]
    Set the sum equal to 1:
    \[
    k + 2k + 2k + 3k + k + k^2 = 1 \implies k^2 + (k + 2k + 2k + 3k + k) = 1 \implies k^2 + 8k = 1.
    \]
    Solve for \(k\):
    \[
    k^2 + 8k - 1 = 0,
    \]
    and find \(k = \frac{1}{10}\). Then,
    \[
    P(X<6) = \left(\frac{1}{10}\right)^2 + \frac{8}{10} = \frac{1}{100} + \frac{80}{100} = \frac{81}{100}.
    \]
    
    \item Find \( P(X=0) \) given that \( P(X=r) \propto \alpha^{r} \) for \( 0 < \alpha < 1 \).
    
    \item What are the possible values of \(X\)?
    
    \item Since there are 16 good oranges and 4 bad oranges, and the oranges are randomly selected, find the probability distribution of \(X\), the number of bad oranges in the group.
    
    \item An die is rolled twice. Let \(X\) denote the number of times six occurs. Find the probability distribution of \(X\).
    
    \item Not occur in the \(i\)th throw. Then, in both throws:
    \[
    P(X=1) = P[(F_1 \text{ and } S_2) \text{ or } (S_1 \text{ and } F_2)] = P(F_1 \cap S_2) + P(S_1 \cap F_2) = P(F_1) P(S_2) + P(S_1) P(F_2) = \frac{5}{6} \times \frac{1}{6} + \frac{1}{6} \times \frac{5}{6} = \frac{10}{36} = \frac{5}{18}
    \]
    and
    \[
    P(X=2) = P(S_1 \cap S_2) = P(S_1) P(S_2) = \frac{1}{6} \times \frac{1}{6} = \frac{1}{36}.
    \]
    The probability distribution of \(X\) is:
    \[
    \begin{tabular}{c c c}
    \(X\) & 0 & 1 & 2 \\
    \(P(X)\) & \(\frac{25}{36}\) & \(\frac{5}{18}\) & \(\frac{1}{36}\)
    \end{tabular}
    \]
    
    \item Draw \(s_i\) independently. Then, the probability \(P(X=0)\) = probability of not getting a king in the two draws:
    \[
    P(\text{not getting a king in 1st and 2nd}) = P(F_1 \cap F_2) = P(F_1) P(F_2) = \frac{48}{52} \times \frac{48}{52} = \frac{144}{169}.
    \]
    The probability \(P(X=1)\) of getting exactly one king:
    \[
    P((S_1 \cap F_2) \text{ or } (F_1 \cap S_2)) = P(S_1) P(F_2) + P(F_1) P(S_2) = \frac{48}{52} \times \frac{4}{52} + \frac{4}{52} \times \frac{48}{52} = \frac{24}{169}.
    \]
    The probability \(P(X=2)\) of getting two kings:
    \[
    P(S_1 \cap S_2) = P(S_1) P(S_2) = \frac{4}{52} \times \frac{4}{52} = \frac{1}{169}.
    \]
    The distribution:
    \[
    \begin{array}{c|c|c|c}
    X & 0 & 1 & 2 \\
    P(X) & \frac{144}{169} & \frac{24}{169} & \frac{1}{169}
    \end{array}
    \]
    
    \item (i) Find the probability distribution of the number of tosses \(X\). \\
    (ii) Verify that the probabilities sum to 1.
    
    \item If \(X\) is the number of red balls in a random draw of three balls, and the balls are drawn without replacement, find the probability distribution of \(X\), where \(X\) can take values \(0,1,2,3\).
    
    \item Calculate \(P(X=0)\), \(P(X=1)\), \(P(X=2)\), and \(P(X=3)\) given the probabilities described.
    
    \item Find the probability distribution of \(X\), the total number of green balls drawn in three draws without replacement.
    
    \item Find the probability of getting a green ball in three draws if the probability of getting no green ball in three draws is \(\frac{5}{8} \times \frac{4}{7} \times \frac{3}{6}\).
    
    \item Calculate the probability of getting exactly one green ball in three draws.
    
    \item Find \(P(X=3)\) given the probabilities of events \(G_1, G_2, G_3\) and their conditional probabilities.
    
    \item Given the probability distribution of \(X\), find \(P(X \leq 1)\).
    
    \item Find the probability that \(X < 1\).
    
    \item Determine the probability distribution of \(X\) given:
    \[
    P(X=0) = \frac{\binom{7}{4}}{\binom{10}{4}} = \frac{1}{6}
    \]
    \[
    P(X=1) = \frac{\binom{3}{1} \times \binom{7}{3}}{\binom{10}{4}} = \frac{1}{2}
    \]
    \[
    P(X=2) = \frac{\binom{3}{2} \times \binom{7}{2}}{\binom{10}{4}} = \frac{3}{10}
    \]
    \[
    P(X=3) = \frac{\binom{3}{3} \times \binom{7}{1}}{\binom{10}{4}} = \frac{1}{30}
    \]
    
    \item Calculate \(P(X \leq 1)\) and verify it equals \(\frac{2}{3}\).
    
    \item Find \(P(X<1)\).
    
    \item Calculate \(P(0 < X < 2)\) given \(P(X=1) = \frac{1}{2}\).
    
    \item A box contains numbers 0, 1, 2, 3 with probabilities:
    \[
    P(X=0) = \frac{1}{8}, \quad P(X=1) = \frac{3}{8}, \quad P(X=2) = \frac{3}{8}, \quad P(X=3) = \frac{1}{8}.
    \]
    Find the probability distribution of \(X\).
    
    \item Find the probability distribution of the number of tails in two coin tosses, assuming the probability of head is \(p\):
    \[
    P(H) = p, \quad P(T) = 1 - p.
    \]
    The probability of tails in one toss is \(1 - p\). The distribution of \(X\), the number of tails in two tosses, is:
    \[
    P(X=0) = P(HH) = p^2,
    \]
    \[
    P(X=1) = P(HT) + P(TH) = 2p(1-p),
    \]
    \[
    P(X=2) = P(TT) = (1-p)^2.
    \]
    
    \item When two trials are independent, the probabilities are:
    \[
    P(X=1) = \frac{3}{8}.
    \]
    Calculate \(P(X=2)\).
    
    \item The probability distribution of \(X\) is:
    \[
    P(X=0) = \frac{9}{16}, \quad P(X=1) = \frac{3}{8}, \quad P(X=2) = \frac{1}{16}.
    \]
    
    \item A die is loaded such that an even number is twice as likely as an odd number. Let \(p\) be the probability of an odd number. Then:
    \[
    P(odd) = p, \quad P(even) = 2p.
    \]
    Since total probability sums to 1:
    \[
    p + 2p + p + 2p + p + 2p = 1,
    \]
    which simplifies to:
    \[
    9p = 1 \implies p = \frac{1}{9}.
    \]
    
    \item Given \(P(1) + P(4) = p + 2p = 3p = \frac{1}{3}\), find \(p\):
    \[
    3p = \frac{1}{3} \implies p = \frac{1}{9}.
    \]
    
    \item The probability that a die does not show a perfect score in both throws is \(\frac{4}{9}\). Find \(P(X=0)\), the probability of zero perfect scores.
    
    \item Calculate the probability \(P(X=1)\), the probability of exactly one perfect score.
    
    \item The probability of perfect scores in both throws:
    \[
    P(X=2) = \frac{1}{9}.
    \]
    \[
    P(X=0) = \frac{4}{9}, \quad P(X=1) = \frac{4}{9}.
    \]
    Verify they sum to 1.
    
    \item In a different die, \(P(6) = \frac{1}{2}\), and the probabilities of other outcomes are equal. Given \(P(1) = \frac{2}{5}\), find \(P(2) = P(3) = P(4) = P(5)\).
    
    \item When two dice are thrown, \(X\) = number of ones seen. Then:
    \[
    P(X=0) = \frac{27}{50},
    \]
    \[
    P(X=1) = \frac{21}{50},
    \]
    \[
    P(X=2) = \frac{2}{50}.
    \]
    
    \item Which of the following probability distributions are valid?
    \begin{enumerate}
        \item \(X: 3, 1, 0, -1\), with probabilities \(0.3, 0.2, 0.4, 0.1\).
        \item \(X: 0, 1, 2\), with probabilities \(0.6, 0.4, 0.2\).
        \item \(X: 0, 1, 2, 3, 4\), with probabilities \(0.1, 0.5, 0.2, 0.1, 0.1\).
        \item \(X: 0, 1, 2, 3\), with probabilities \(0.3, 0.2, 0.4, 0.1\).
    \end{enumerate}
    
    \item A random variable \(X\) has the distribution:
    \[
    \begin{array}{|c|c|c|c|c|c|c|}
    \hline
    \text{Values of } X & -2 & -1 & 0 & 1 & 2 & 3 \\
    \hline
    P(X) & 0.1 & k & 0.2 & 2k & 0.3 & k \\
    \hline
    \end{array}
    \]
    Find \(k\).
    
    \item The probability distribution function:
    \[
    \begin{array}{l l l l l l l l l l l}
    \text{Values of } X : & 0 & & 1 & & 2 & & 3 & & 4 & & 5 & 6 & 7 & 8 \\
    P(X) : & a & & 3a & & 5a & & 7a & & 9a & & 11a & & 13a & & 15a & & 17a
    \end{array}
    \]
    Find the value of \(a\), \(P(X<3)\), \(P(X \geq 3)\), \(P(0 < X < 5)\).

    \item The probability distribution:
    \[
    \begin{array}{|c|c|c|c|c|c|c|}
    \hline
    x_i & 0 & 1 & 2 \\
    \hline
    p_i & 3 c^3 & 4 c - 10 c^2 & 5 c - 1 \\
    \hline
    \end{array}
    \]
    where \(c > 0\). Find \(P(X<2)\) and \(P(1 < X \leq 2)\).

    \item A random variable \(X\) takes values \(x_1, x_2, x_3, x_4\). Given:
    \[
    2 P(X=x_1) = 3 P(X=x_2) = P(X=x_3) = 5 P(X=x_4).
    \]
    Find the distribution of \(X\).

    \item The values of \(X\) are 0, 1, 2, 3 such that:
    \[
    2 P(X=0) = P(X>0) = P(X<0); \quad P(X=-3) = P(X=-2) = P(X=-1); \quad P(X=1) = P(X=2) = P(X=3).
    \]
    Obtain the distribution of \(X\).

    \item Find the distribution of the number of heads when three coins are tossed.

    \item Find the distribution of the number of red balls when two balls are drawn at random from a bag containing 2 red and 3 blue balls.

    \item Five defective and five good items are mixed. Find the distribution of the number of defective items drawn at random.

    \item Assuming all outcomes are equally likely, what is the distribution of the number of defective items in a sample of two?

    \item A student is selected at random from a class of 50 students, and their age \(X\) is recorded. Find the probability distribution of \(X\).

    \item Five defective bolts are mixed with twenty good ones. If four bolts are drawn at random, find the distribution of the number of defective bolts.

    \item Find the distribution of the number of aces in a draw of cards.

    \item Two cards are drawn successively without replacement from a well-shuffled pack of 52 cards. Find the distribution of the number of aces.

    \item Three cards are drawn successively with replacement from a well-shuffled deck. Let \(X\) be the number of hearts in the three cards. Find the distribution of \(X\).

    \item An urn contains 4 red and 3 blue balls. Find the distribution of the number of blue balls drawn in a draw of two balls.

    \item Two cards are drawn simultaneously from a well-shuffled deck. Find the distribution of the number of successes when getting a spade is considered a success.

    \item A fair die is tossed twice. If the number on the top is less than 3, it is a success. Find the distribution of the number of successes.

    \item \(X\) is the number of black balls. What are the possible values of \(X\)?

    \item Let \(X\) be the difference between the number of heads and tails when a coin is tossed 6 times. What are the possible values of \(X\)?

    \item From a lot of 10 bulbs, which includes 3 defective, a sample of 2 bulbs is drawn. Find the distribution of the number of defective bulbs.

    \item (i) Determine the value of \(k\).

    \item (ii) Determine \(P(X \leq 2)\) and \(P(X > 2)\).

    \item (iii) Find \(P(X \leq 2) + P(X > 2)\).

    \item (31) A bag contains 19 tickets numbered 1 to 19. A ticket is drawn at random and then another without replacement. Find the distribution of \(X\).

    \begin{enumerate}
        \item[(a)] Find the distribution of \(X\).
        \item[(b)] Compute the expectation \(E[X]\) and variance \(\operatorname{Var}(X)\).
    \end{enumerate}
    
    \item Given the distribution:
    \[
    P(X=0) = \frac{969}{2530},
    \]
    \[
    P(X=2/3/8) = \frac{2}{38},
    \]
    \[
    P(X=4/253) = \frac{1}{2530}.
    \]
    Find the distribution of \(X\).

    \item The distribution:
    \[
    P(X=11/4/2530) = \frac{969}{2530},
    \]
    \[
    P(X=2/3/8) = \frac{2}{38},
    \]
    \[
    P(X=4/253) = \frac{1}{2530}.
    \]
    (Note: The exact values and context seem inconsistent; please clarify or specify the intended distribution.)

    \item The distribution:
    \[
    P(X=0) = \frac{7}{15}, \quad P(X=1) = \frac{7}{15}, \quad P(X=2) = \frac{1}{15}.
    \]
    \end{enumerate}
\end{enumerate}

\end{document}